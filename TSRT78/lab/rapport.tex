\documentclass[10pt]{article}

\usepackage{times}
\usepackage{mathptmx}
\usepackage{amsmath}
\usepackage{mathtools}
\usepackage{graphicx}
\usepackage{epstopdf}

\setlength\parindent{0pt}


\raggedbottom
\sloppy

\title{Fundamental Signal Processing\\
\emph{TSRT78}}

%\author{David Habrman \\ davha227, 920908-2412\\
%Jens Edhammer \\ jened502, 920128-5112 }

%\date{\today}

\begin{document}

\maketitle

\newpage
\tableofcontents
\newpage

\section{Introduction}
This is a laboration in the course "Digital Signal Processing", TSRT78.
It consists of three parts, whistle, vowel and speech encoding as in GSM.

The first assignment, whistle, is to try to whistle a sine and then
investigate how pure the whistle actually is.

In part two, vowel, the assignment is to model the sound of a vowel
using an AR model. A proper order of the model was to be determined.
The sound was then simulated using a suitable input signal and a
model of proper order.

The last assignment, speech encoding as in GSM, is to encode speech
using GSM and investigate the result.

\subsection{Notation}
$f_s$ - sample frequency\\
$N$ - number of samples

\section{Whistle}
Two seconds of a whistle was recorded at 8 $kHz$, $x[n]$, which gives 16000 samples.
This sound was then filtered to get rid of unwanted frequencies, $x_{filt}[n]$.
The energy of the filtered, $E_{filt}$, and unfiltered sound, $E_{tot}$,
was caluculated in the time domain using equation~\ref{eq:Etimefilt}
and~\ref{eq:Etime}. The energies was also calculated in the frequency
domain using equation ~\ref{eq:Efeqfilt} and~\ref{eq:Efeq}.

\begin{equation}
  \label{eq:Etimefilt}
  E_{filt} =\sum\limits_{n=-\infty}^{N-1} |x_{filt}[n]|^2
\end{equation}

\begin{equation}
  \label{eq:Etime}
  E_{tot} =\sum\limits_{n=-\infty}^{N-1} |x[n]|^2
\end{equation}

\begin{equation}
  \label{eq:Efeqfilt}
  E_{filt} =\frac{1}{N}\sum\limits_{k=0}^{N-1} |X_{filt}[k]|^2
\end{equation}

\begin{equation}
  \label{eq:Efeq}
  E_{tot} =\frac{1}{N}\sum\limits_{k=0}^{N-1} |X[k]|^2
\end{equation}

\subsection{Method}

\subsection{Result and conclusion}

\section{Vowel}
\subsection{Method}
\subsection{Results and conclusion}

\section{Speech encoding as in GSM}
\subsection{Method}
\subsection{Results and conclusion}

\end{document}
