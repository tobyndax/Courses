\documentclass[10pt]{article}

\usepackage{times}
\usepackage{mathptmx}
\usepackage{amsmath}
\usepackage{mathtools}
\usepackage{graphicx}
\usepackage{epstopdf}
\usepackage{listings}
\usepackage{color} %red, green, blue, yellow, cyan, magenta, black, white
\definecolor{mygreen}{RGB}{28,172,0} % color values Red, Green, Blue
\definecolor{mylilas}{RGB}{170,55,241}


\setlength\parindent{0pt}


\raggedbottom
\sloppy

\title{Fundamental Signal Processing\\
\emph{TSRT78}}

%\author{David Habrman \\ davha227, 920908-2412\\
%Jens Edhammer \\ jened502, 920128-5112 }

%\date{\today}

\begin{document}

  \lstset{language=Matlab,%
    %basicstyle=\color{red},
    breaklines=true,%
    morekeywords={matlab2tikz},
    keywordstyle=\color{blue},%
    morekeywords=[2]{1}, keywordstyle=[2]{\color{black}},
    identifierstyle=\color{black},%
    stringstyle=\color{mylilas},
    commentstyle=\color{mygreen},%
    showstringspaces=false,%without this there will be a symbol in the places where there is a space
    numbers=left,%
    numberstyle={\tiny \color{black}},% size of the numbers
    numbersep=9pt, % this defines how far the numbers are from the text
    emph=[1]{for,end,break},emphstyle=[1]\color{red}, %some words to emphasise
    %emph=[2]{word1,word2}, emphstyle=[2]{style},
}


\maketitle

\newpage
\tableofcontents
\newpage

\section{Introduction}
This is a laboration in the course "Digital Signal Processing", TSRT78.
It consists of three parts, whistle, vowel and speech encoding as in GSM.

The first assignment, whistle, is to try to whistle a sine and then
investigate how pure the whistle actually is.

In part two, vowel, the assignment is to model the sound of a vowel
using an AR model. A proper order of the model was to be determined.
The sound was then simulated using a suitable input signal and a
model of proper order.

The last assignment, speech encoding as in GSM, is to encode speech
using GSM and investigate the result.

\subsection{Notation}
$f_s$ - sample frequency\\
$N$ - number of samples

\section{Whistle}
Two seconds of a whistle was recorded at 8 $kHz$, $x[n]$, which gives 16000 samples.
This sound was then filtered to get rid of unwanted frequencies, $x_{filt}[n]$.
The energy of the filtered, $E_{filt}$, and unfiltered sound, $E_{tot}$,
was caluculated in the time domain using equation~\ref{eq:Etimefilt}
and~\ref{eq:Etime}. The energies was also calculated in the frequency
domain using equation ~\ref{eq:Efeqfilt} and~\ref{eq:Efeq}.

\begin{equation}
  \label{eq:Etimefilt}
  E_{filt} =\sum\limits_{n=-\infty}^{N-1} |x_{filt}[n]|^2
\end{equation}

\begin{equation}
  \label{eq:Etime}
  E_{tot} =\sum\limits_{n=-\infty}^{N-1} |x[n]|^2
\end{equation}

\begin{equation}
  \label{eq:Efeqfilt}
  E_{filt} =\frac{1}{N}\sum\limits_{k=0}^{N-1} |X_{filt}[k]|^2
\end{equation}

\begin{equation}
  \label{eq:Efeq}
  E_{tot} =\frac{1}{N}\sum\limits_{k=0}^{N-1} |X[k]|^2
\end{equation}

The purity of the whistle, $P$, was calculated using equation~\ref{eq:Purity}.
It was calculated using energies from the time domain but also the
frequency domain.

\begin{equation}
  \label{eq:Purity}
  P =1-\frac{E_{filt}}{E_{tot}}
\end{equation}

An AR model of order two, AR(2) was calculated. A purity measure
is computed by calculating the distance from each poole to the unit
circle. This is a good purity measure since a pure sinusoid have
two pooles on the unit circle.

Finally the dominating frequency of the whistle was estimated.
This was done using the spectrum of the non-parametric method
as well as the spectrum of the parametric method.

\subsection{Result and conclusion}
The energies where calculated to $E_{tot}=1.1064*10^3$, $E_{filt}=1.1033*10^3$,
$E_{Tot}=1.1064*10^3$ and $E_{Filt}=1.1033*10^3$ which gives the harmonic distortion,
$P= 0.0029$ in both cases. This is the dominant frequency.

The non-parametric spectrum of the whistle can be seen in figure~\ref{fig:Wfilt}
and the parametric spectrum in figur~\ref{fig:WAR}. Both methods give the same peak
at $0.96 rad/s$.


\begin{figure}[!hp]

    \begin{center}
      \includegraphics[width=0.6\textwidth]{Wfilt}
    \caption{Non-parametric spectrum \label{fig:Wfilt}}
    \end{center}

\end{figure}

\begin{figure}[!hp]

    \begin{center}
      \includegraphics[width=0.6\textwidth]{WAR}
    \caption{Parametric spectrum \label{fig:WAR}}
    \end{center}

\end{figure}


\section{Vowel}
Two seconds of sound of the two vowels a and o was recorded at
8 $kHz$. These sounds were then used to compute AR models of
different orders. To determine which order is required a couple of validation
methods were applied. Firstly the mean of the squared prediction errors were compared for different
orders. A distinct improvment can be noticed between order 1 and 2, with some
significant improvement up untill order 5 for the o-sound, which can be seen in
figure~\ref{fig:oLeLv}. For the a-sound, in figure~\ref{fig:aLeLv}, we see that
 we get improvements of significance up until order 10.


\begin{figure}[!hp]

    \begin{center}
      \includegraphics[width=0.6\textwidth]{oLeLv}
    \caption{O-sound mean squared prediction error \label{fig:oLeLv}}
    \end{center}

\end{figure}

\begin{figure}[!hp]

    \begin{center}
      \includegraphics[width=0.6\textwidth]{aLeLv}
    \caption{A-sound mean squared prediction error \label{fig:aLeLv}}
    \end{center}

\end{figure}

The bodeplots for both the a-sound and the o-sound for the different orders of
the systems are compared with bodeplots of the data. The a-sound presented in
figure~\ref{fig:aBode}, shows that both order 10 and 15 follow the real data
rather good. For o-sound, in figure~\ref{fig:oBode}, we see that already at
5th order we get a decent fit.


\begin{figure}[!hp]

    \begin{center}
      \includegraphics[width=0.8\textwidth]{aBode}
    \caption{a-sound, bodeplot \label{fig:aBode}}
    \end{center}

\end{figure}

\begin{figure}[!hp]

    \begin{center}
      \includegraphics[width=0.8\textwidth]{oBode}
    \caption{o-sound, bodeplot \label{fig:oBode}}
    \end{center}
\end{figure}

\subsection{Results and conclusion}
For the o-sound AR(5) was used and for the a-sound AR(10) was used.
The results are decent when listened to. You can easily discern the a from the o.
Both the a-sound and the o-sound sound very monotone, with some improvements if
the system orders are increased.
Examining the fourier transform of the original signal and the reconstructed
signal we see that the result does match decently, see figure~\ref{fig:ffto} and
figure~\ref{fig:ffta}.


\begin{figure}[!hp]

    \begin{center}
      \includegraphics[width=0.6\textwidth]{aFFT}
    \caption{Fouriertransform of o-sound and reconstructed o-sound \label{fig:ffto}}
    \end{center}

\end{figure}

\begin{figure}[!hp]

    \begin{center}
      \includegraphics[width=0.6\textwidth]{oFFT}
    \caption{Fouriertransform of a-sound and reconstructed a-sound \label{fig:ffta}}
    \end{center}

\end{figure}



\section{Speech encoding as AR-model}
The sentence "The clever fox surprised the rabbit" was recorded at
8 $kHz$. This sound was the subject for a simplified GSM encoding.
The sound was divided into 150 segments of 160 samples each. A
AR model of order 8 was estimated for each segment. The sound was
then simulated using a specific pulse train for each segment as
input signal. Each pulse period, $D$, and amplitute, $A$, was
estimated by taking the maximum of the covariance function of the
residuals. Each pulse train is as long as a segment, 160 samples.


\subsection{Results and conclusion}
The original sound, the reconstructed sound using a pulse train with
dynamic amplitude and the reconstructed sound using a pulse train
with constant amplitude one can be seen in figure~\ref{fig:recon}.

\begin{figure}[!hp]

    \begin{center}
      \includegraphics[width=0.6\textwidth]{recon}
    \caption{Parametric spectrum \label{fig:recon}}
    \end{center}

\end{figure}

The original sound and the reconstructed sound using a pulse train
with dynamic amplitude have the same major appearance as can be seen in
figure~\ref{fig:recon}. The sound reconstructed with a pulse train with
constant amplitude however is quite bad.

By listening to the sound reconstructed using different model orders
we draw the conclusion that orders lower than $8$ give unsatesfactory
results and orders higher than $8$ give almost no improvement.

For the reciever to reconstruct the sound all model coefficients,
except for the one set to one, periodtime and segment amplitude.
In our case it means that an AR(8) need to send at least 10 parameters
for each segment down from 160. This gives a compression by a
factor of 16.

\end{document}
