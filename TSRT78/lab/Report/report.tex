\documentclass[10pt]{article}

\usepackage{times}
\usepackage{mathptmx}
\usepackage{amsmath}
\usepackage{mathtools}
\usepackage{graphicx}
\usepackage{epstopdf}
\usepackage{listings}
\usepackage{color} %red, green, blue, yellow, cyan, magenta, black, white
\definecolor{mygreen}{RGB}{28,172,0} % color values Red, Green, Blue
\definecolor{mylilas}{RGB}{170,55,241}


\setlength\parindent{0pt}


\raggedbottom
\sloppy

\title{Fundamental Signal Processing\\
\emph{TSRT78}}

%\author{David Habrman \\ davha227, 920908-2412\\
%Jens Edhammer \\ jened502, 920128-5112 }

%\date{\today}

\begin{document}

  \lstset{language=Matlab,%
    %basicstyle=\color{red},
    breaklines=true,%
    morekeywords={matlab2tikz},
    keywordstyle=\color{blue},%
    morekeywords=[2]{1}, keywordstyle=[2]{\color{black}},
    identifierstyle=\color{black},%
    stringstyle=\color{mylilas},
    commentstyle=\color{mygreen},%
    showstringspaces=false,%without this there will be a symbol in the places where there is a space
    numbers=left,%
    numberstyle={\tiny \color{black}},% size of the numbers
    numbersep=9pt, % this defines how far the numbers are from the text
    emph=[1]{for,end,break},emphstyle=[1]\color{red}, %some words to emphasise
    %emph=[2]{word1,word2}, emphstyle=[2]{style},
}


\maketitle

\newpage
\tableofcontents
\newpage

\section{Introduction}
This is a laboration in the course "Digital Signal Processing", TSRT78.
It consists of three parts, whistle, vowel and speech encoding as in GSM.

The first assignment, whistle, is to try to whistle a sine and then
investigate how pure the whistle actually is.

In part two, vowel, the assignment is to model the sound of a vowel
using an AR model. A proper order of the model was to be determined.
The sound was then simulated using a suitable input signal and a
model of proper order.

The last assignment, speech encoding as in GSM, is to encode speech
using GSM and investigate the result.

\subsection{Notation}
$f_s$ - sample frequency\\
$N$ - number of samples

\section{Whistle}
Two seconds of a whistle was recorded at 8 $kHz$, $x[n]$, which gives 16000 samples.
This sound was then filtered to get rid of unwanted frequencies, $x_{filt}[n]$.
The energy of the filtered, $E_{filt}$, and unfiltered sound, $E_{tot}$,
was caluculated in the time domain using equation~\ref{eq:Etimefilt}
and~\ref{eq:Etime}. The energies was also calculated in the frequency
domain using equation ~\ref{eq:Efeqfilt} and~\ref{eq:Efeq}.

\begin{equation}
  \label{eq:Etimefilt}
  E_{filt} =\sum\limits_{n=-\infty}^{N-1} |x_{filt}[n]|^2
\end{equation}

\begin{equation}
  \label{eq:Etime}
  E_{tot} =\sum\limits_{n=-\infty}^{N-1} |x[n]|^2
\end{equation}

\begin{equation}
  \label{eq:Efeqfilt}
  E_{filt} =\frac{1}{N}\sum\limits_{k=0}^{N-1} |X_{filt}[k]|^2
\end{equation}

\begin{equation}
  \label{eq:Efeq}
  E_{tot} =\frac{1}{N}\sum\limits_{k=0}^{N-1} |X[k]|^2
\end{equation}

The purity of the whistle, $P$, was calculated using equation~\ref{eq:Purity}.
It was calculated using energies from the time domain but also the
frequency domain.

\begin{equation}
  \label{eq:Purity}
  P =1-\frac{E_{filt}}{E_{tot}}
\end{equation}

An AR model of order two, AR(2) was estimated on both the filtered
%måste göra klart harmonic dist med AR-model

Finally the dominating frequency of the whistle was estimated.
This was done using the spectrum of the non-parametric method
as well as the spectrum of the parametric method.

\subsection{Result and conclusion}

\section{Vowel}
Two seconds of sound of the two vowels a and o was recorded at
8 $kHz$. These sounds were then used to compute AR models of
different orders.Three validation methods are used to determine
a proper order for the two AR models.

The first method is to compare model predictions with real data.
By doing this a model order that is good enough can be choosen.
Another method is to compare the non-parametric spectrum estimate
with the spectrum of the AR models of different orders. A order
that is good enough can again be choosen. The third and final
validation method is to estimate the covatiance functions of the
residuals and again choose a good enough model.

Finally the two AR models of proper orders were simulated. Since
the two sounds are periodic the simulation is done by using a
pulse train as input. The pulse trains have the same period as the
sound of the vowels they are used to simulate.

\subsection{Results and conclusion}

\section{Speech encoding as AR-model}
The sentence "The clever fox surprised the rabbit" was recorded at
8 $kHz$. This sound was the subject for a simplified GSM encoding.
The sound was divided into 150 segments of 160 samples each. A
AR model of order 8 was estimated for each segment. The sound was
then simulated using a specific pulse train for each segment as
input signal. Each pulse period, $D$, and amplitute, $A$, was
estimated by taking the maximum of the covariance function of the
residuals. Each pulse train is as long as a segment, 160 samples.



\subsection{Results and conclusion}

\end{document}
